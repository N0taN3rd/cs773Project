\documentclass[letterpaper,10pt]{article}
\usepackage[pdftex]{graphicx}
\usepackage{listings}
\usepackage{alltt}
\usepackage{color}
\usepackage{mathtools}
\usepackage{hyperref}
\usepackage{caption}
\usepackage{pgfplotstable}
\usepackage{tabu}
\usepackage[margin=0.5in]{geometry}
\usepackage{float}
\usepackage{xcolor}
\usepackage{makeidx}
\usepackage[toc,page]{appendix}
\usepackage{cprotect}



\captionsetup{labelformat=empty}

\pgfplotsset{compat=1.13}


\colorlet{punct}{red!60!black}
\definecolor{background}{HTML}{EEEEEE}
\definecolor{delim}{RGB}{20,105,176}
\colorlet{numb}{magenta!60!black}


\definecolor{dkgreen}{rgb}{0,0.6,0}
\definecolor{gray}{rgb}{0.5,0.5,0.5}
\definecolor{mauve}{rgb}{0.58,0,0.82}

\hypersetup{
    colorlinks,
    citecolor=black,
    filecolor=black,
    linkcolor=black,
    urlcolor=black
}
\definecolor{lightgray}{rgb}{.9,.9,.9}
\definecolor{darkgray}{rgb}{.4,.4,.4}
\definecolor{purple}{rgb}{0.65, 0.12, 0.82}

\makeindex

\lstset{ %
  language=R,                     % the language of the code
  basicstyle=\footnotesize,       % the size of the fonts that are used for the code
  numbers=left,                   % where to put the line-numbers
  numberstyle=\tiny\color{gray},  % the style that is used for the line-numbers
  stepnumber=1,                   % the step between two line-numbers. If it's 1, each line
                                  % will be numbered
  numbersep=5pt,                  % how far the line-numbers are from the code
  backgroundcolor=\color{white},  % choose the background color. You must add \usepackage{color}
  showspaces=false,               % show spaces adding particular underscores
  showstringspaces=false,         % underline spaces within strings
  showtabs=false,                 % show tabs within strings adding particular underscores
  frame=single,                   % adds a frame around the code
  rulecolor=\color{black},        % if not set, the frame-color may be changed on line-breaks within not-black text (e.g. commens (green here))
  tabsize=2,                      % sets default tabsize to 2 spaces
  captionpos=b,                   % sets the caption-position to bottom
  breaklines=true,                % sets automatic line breaking
  breakatwhitespace=false,        % sets if automatic breaks should only happen at whitespace
  title=\lstname,                 % show the filename of files included with \lstinputlisting;
                                  % also try caption instead of title
  keywordstyle=\color{blue},      % keyword style
  commentstyle=\color{dkgreen},   % comment style
  stringstyle=\color{mauve},      % string literal style
  escapeinside={\%*}{*)},         % if you want to add a comment within your code
  morekeywords={*,...}            % if you want to add more keywords to the set
} 

\begin{document} 

\begin{titlepage}

\begin{center}
\Huge{Course Project}

\Large{CS773: Data Mining and Security}

\Large{Summer 2016}

\Large{John Berlin, Yun Han}
\end{center}

\end{titlepage}

\printindex
\tableofcontents
\listoftables
\listoffigures
\lstlistoflistings

\newpage
\addcontentsline{toc}{section}{Data Changes}
\section*{Data Changes}
For questions two through four the data was reduced to \textit{cuisine,atmosphere,occasion,price,style}. Any restaurant that did not have one of those features the fake feature none was added and for those that 
had multiple features per feature category i.e one atmosphere, cuisine and price but two occasion features with non style the following was done. This is shown in table \hyperref[tab:dchange]{\ref{tab:dchange}}.
\begin{table}[h]
\caption{Data change example}
\begin{tabu}{ccccc} 
cuisine & atmosphere & occasion & price & style \\
Italian & Excellent Decor & Open on Sundays & \$15-\$30 & none\\
Italian & Excellent Decor & Open on Mondays & \$15-\$30 & none \\
\end{tabu}

 \label{tab:dchange} 
\end{table} 


\addcontentsline{toc}{section}{Task i}
\section*{Task i}
\index{Task i}
\begin{verbatim}
Study the provided features and classify them into one of the standard (cuisine, style,
price, atmosphere, and occasion) or into your own created additional categories. Limit
the new categories to at most 5. If you think that a feature fits into more than one of
your categories, do put them in all the categories that they fit in. This could be more
an exception than a role. Typically, there should be a 1-1 mapping of features to
categories.
\end{verbatim}
\subsection*{Results}

\begin{table}[h]
\centering
\caption{Atmosphere Classification}
\begin{tabu}{ccc} 
An Historic Spot & An Out Of The Way Find & Authentic \\
Buffet Dining & Business Scene & Cafe/Garden Dining \\
Classic Hotel Dining & Creative & Credit cards are not accepted \\
Excellent Decor & Excellent Food & Excellent Service \\
Extraordinary Decor & Extraordinary Food & Extraordinary Service \\
Fabulous Views & Fabulous Wine Lists & Fair Decor \\
Fair Food & Fair Service & Focus on Dessert \\
For the Young and Young at Heart & Good Decor & Good Food \\
Good Out of Town Business & Good Service & Good for Younger Kids \\
Great for People Watching & Health Conscious Menus & Hip Place To Be \\
Little Known But Well Liked & Near-perfect Decor & Near-perfect Food \\
Near-perfect Service & Need To Dress & No Liquor Served \\
No Reservations & No Smoking Allowed & Old World Cafe Charm \\
On the Beach & Parking/Valet & People Keep Coming Back \\
Place for Singles & Poor Decor & Pub Feel \\
Quiet for Conversation & Quirky & Relaxed Senior Scene \\
Romantic & Singles Scene & Tourist Appeal \\
Up and Coming & Very Busy - Reservations a Must & Warm spots by the fire \\
Wheelchair Access & & \\
\end{tabu}
\end{table}

\begin{table}[h]
\centering
\caption{Cuisine Classification}
\begin{tabu}{ccccc}
Afghanistan & African & American & Argentinean & Armenian \\
Asian & Austrian & Bar-B-Q & Belgian & Brazilian \\
Burmese & Burritos & Cajun & Cambodian & Canadian \\
Caribbean & Chinese & Coffee and Dessert & Creole & Cuban \\
Czech & Dim Sum & Egyptian & Ethiopian & Filipino \\
Fountain and Ice Cream & French & German & Greek & Guatemalan \\
Hamburgers & Beer \& Hot Dogs & Hungarian & Indian & Indonesian \\
Irish & Italian & Jamaican & Japanese & Jewish \\
Korean & Latin & Lebanese & Malaysian & Mediterranean \\
Mexican & Moroccan & Nicaraguan & Pacific New Wave & Pacific Rim \\
Persian & Peruvian & Polish & Polynesian & Portuguese \\
Puerto Rican & Romanian & Roumanian & Russian & Salvadoran \\
Scandinavian & Seafood & Spanish & Sushi & Swiss \\
Tapas & Tex Mex & Tex-Mex & Thai & Tibetan \\
Tunisian & Turkish & Ukrainian & Ukranian & Vegetarian \\
Venezuelan & Vietnamese & Wine and Beer & Yugoslavian &  \\
\end{tabu}
\end{table}

\begin{table}[h]
\centering
\caption{Occasion Classification}
\begin{tabu}{cccc}
After Hours Dining & Catering for Special Events & Dancing & Delivery Available \\
Dining After the Theater & Dining Outdoors & Early Dining & Entertainment \\
Fine for Dining Alone & Game & Great Place to Meet for a Drink & Happy Hour \\
Late Night Menu & Long Drive & Margaritas & Menus in Braille \\
Open for Breakfast & Open on Mondays & Open on Sundays & Other Quick Food \\
Parties and Occasions & Picnics & Pre-theater Dining & Private Parties \\
Private Rooms Available & Prix Fixe Menus & See the Game & Short Drive \\
Special Brunch Menu & Takeout Available & Walk & Weekend Brunch \\
Weekend Dining & Weekend Jazz Brunch & Weekend Lunch &  \\
\end{tabu}
\end{table}

\begin{table}[h]
\centering
\caption{Style Classification}
\begin{tabu}{cccc}
A & American (Contemporary) & American (New) & American (Regional) \\
American (Traditional) & Bakeries & Brasserie & Cab \\
Cafe/Espresso Bars & Cafeterias & Californian & Carry in Wine and Beer \\
Caviar & Central & Coffee Houses & Continental \\
Haute Creole & Haute New Orleans & Coffeehouses & Coffee Shops \\
Deli & Diners & Down-Home & Eastern European \\
Eclectic & Down-Home Creole & Southern & Southwestern \\
South American & Southeast Asian & English & Fast Food \\
Fondue & Franco-Russian & Frankfurters & French Bistro \\
French Classic & French Contemporary & French-Japanese & French (New) \\
French Nouvelle & Grills & Hamburgers & Health Food \\
High Tea & International & Italian (North & Italian (Northern) \\
Italian (North \& South) & Italian Nuova Cucina & Italian (Southern) & Kosher \\
Lithuanian & Middle Eastern & N & Noodle Houses \\
Noodle Shops & Omelettes & Oyster Bars & Pancakes \\
Pastries & Pastry Shops & Pizza & Pizzerias \\
Po' Boys & Power Brokers & Scottish & Soul Food \\
Soulfood & Southern Comfort & Steakhouses & Swiss-French \\
Tacos & Traditional & Yogurt Bar & \\
\end{tabu}
\end{table}

\begin{table}[h]
\centering
\caption{Price Classification}
\begin{tabu}{cccc}
\$15-\$30 & \$30-\$50 & below \$15 & over \$50 \\
\end{tabu}
\end{table}
\clearpage
\newpage
\addcontentsline{toc}{section}{Task ii}
\section*{Task ii}
\index{Task ii}
\begin{verbatim}
Analyze and form rules to characterize the following five types of cuisine: (i) Indian
(ii) Mexican (iii) Italian (iv) French (v) American. For each category, derive a set of
rules based on data available from all cities.
\end{verbatim}
\subsection*{Results}
The results for task ii were generated from an R script seen in listing \hyperref[lst:tii]{\ref{lst:tii}} which uses the C50 decision tree algorithm to produce rules. Data used in this script was generated using a python file \verb+rules.py+ which accompanies this report. The classification was aided through using boosting which is indicated by setting the \textit{trialNum} to ten as well as setting predictor winnowing(feature selection) to true. The ruleGen file can be seen in this listing \hyperref[lst:apendixRF]{\ref{lst:apendixRF}} in the appendices for this report. We also set the flag of rules to true in order to decompose the tree into the bare rules. \newline\newline
Overall the classification of the data the features \textit{price}, \textit{style}, \textit{occasion} were heavily used when generating the rules as seen in table \hyperref[tab:tii_au]{\ref{tab:tii_au}}. Boosting resulted in a \verb+38.7%+ error margin with trials 0, 7 and 8 resulting in least error. The entire rule set for each trial can be found in the files \textit{c50\_10Trial(2).txt} which accompanies this report.
Trial 0 generated the most number of rules which was 118 with the average number of rules being between 40-60. We omit listing the rules here as the file is 5977 lines long but include notable rules as seen below. This is a sampling seen in figure \hyperref[fig:c50]{\ref{fig:c50}} is taken from all trials.
\begin{figure}[h]
\begin{verbatim}
Rule 0/24: (582, lift 6.4)
	style in {Bakeries, Brasserie, Caviar}
	->  class French  [0.998]

Rule 0/38: (3, lift 5.1)
	atmosphere = Romantic
	occasion = After Hours Dining
	price = $30-$50
	style = none
	->  class French  [0.800]
	
Rule 0/58: (4, lift 56.3)
	atmosphere in {Good Food, Good Service}
	price = $15-$30
	style = Carry in Wine and Beer
	->  class Indian  [0.833]

Rule 0/106: (8, lift 11.5)
	occasion in {Dining Outdoors, Late Night Menu}
	price = below $15
	style = Carry in Wine and Beer
	->  class Mexican  [0.900]
	
Rule 3/4: (1984.1/437.8, lift 1.8)
	occasion in {Dancing, Delivery Available, Dining After the Theater,
                     Early Dining, Fine for Dining Alone,
                     Great Place to Meet for a Drink, Long Drive,
                     Menus in Braille, Open on Mondays, Open on Sundays,
                     Picnics, See the Game, Short Drive, Special Brunch Menu,
                     Takeout Available, Weekend Dining}
	style in {Coffee Shops, Eclectic, High Tea, Omelettes, Pastries,
                  Steakhouses}
	->  class American  [0.779]
\end{verbatim}
\caption{Sampling of rules from C50}
\label{fig:c50} 
\end{figure}


\begin{table}[H]
\centering
\setlength\tabcolsep{2pt}
\begin{minipage}{0.38\textwidth}
\caption{Task ii Classification Errors}
\begin{verbatim}
Evaluation on training data (61063 cases):
Trial	        Rules     
-----	  ----------------
	    No      Errors

   0	   119 22584(37.0%)
   1	    60 25297(41.4%)
   2	    61 24058(39.4%)
   3	    45 25904(42.4%)
   4	    61 24293(39.8%)
   5	    53 25591(41.9%)
   6	    43 25873(42.4%)
   7	    78 23947(39.2%)
   8	    65 23669(38.8%)
   9	    52 24269(39.7%)
boost	      23614(38.7%) 
\end{verbatim}
\label{tab:tii_err} 
\end{minipage}%
\begin{minipage}{0.68\textwidth}
\caption{Task ii Classification Confusion Matrix} 
\begin{verbatim}
 (a)   (b)   (c)   (d)   (e)    <-classified as
	  ----  ----  ----  ----  ----
	 25111  1206        2420   360    (a): class American
	  4754  3533        1220    10    (b): class French
	   640    15         229    20    (c): class Indian
	  8781   572        7247   150    (d): class Italian
	  2769               468  1558    (e): class Mexican
\end{verbatim}
 \label{tab:tii_cm} 
\end{minipage} 
\begin{minipage}{0.48\textwidth}
\centering
\caption{Task ii Attribute Usage} 
\begin{verbatim}
	100.00%	price
	100.00%	style
	 99.99%	occasion
	 89.66%	atmosphere
\end{verbatim}
 \label{tab:tii_au} 
\end{minipage}
\end{table}
\clearpage
\newpage
\addcontentsline{toc}{section}{Task iii}
\section*{Task iii}
\index{Task iii}
\begin{verbatim}
Let us now concentrate specifically on the quality of the food. This is specified
through features 73-78. Assuming that the outcome you are interested is in one of the
following categories, determine if there is any relationship between type of cuisine
and the quality indicator. Categories to be considered are: Fair, Good, Excellent.
You can combine the given 6 categories into these 3 categories.
\end{verbatim}
\subsection*{Results}

\clearpage
\newpage
\addcontentsline{toc}{section}{Task iv}
\section*{Task iv}
\index{Task iv}
\begin{verbatim}
Derive association rules among the given features. In other words, does Creative
atmosphere imply a specific category? In particular, experiment with the following
associations:
a. Cuisine and atmosphere
b. Price and atmosphere
c. Price and style
d. Cuisine and occasion.
e. Décor and Price
\end{verbatim}
\subsection*{Results}
\clearpage
\newpage
\addcontentsline{toc}{section}{Task v}
\section*{Task v}
\index{Task v}
\begin{verbatim}
Find an association between a restaurant offering vegetarian (243) to its price and
cuisine.
\end{verbatim}
\subsection*{Results}
\begin{enumerate}
\item Assumptions 
\begin{enumerate}
\item The cuisine types are divided manually and are most of the origination of the food type.
\item The restaurants that do not have vegetarian (243) feature are considered not offering vegetarian
\item The first cuisine type encountered in the restaurant data is extracted for association rule mining.
\item Weka is used to mine the association rules between the three attributes: cuisine, price and vegietarian
\end{enumerate} 
\end{enumerate}
The Tertius method works well to mine the association rules for vegetarian restaurant. Below in figure \hyperref[fig:tv]{\ref{fig:tv}} please find the results from Weka. The default scheme was used (weka.associations.Tertius -K 10 -F 0.0 -N 1.0 -L 4 -G 0 -c 0 -I 0 -P 0). The perl code used for the task is seen in listing \hyperref[lst:v]{\ref{lst:v}}.
\begin{figure}[h]
\begin{verbatim}
 1. /* 0.098122 0.017548 */ price = 162 ==> vegie = Yes or cuisine = 221
 2. /* 0.094522 0.016346 */ price = 162 ==> vegie = Yes or cuisine = 229
 3. /* 0.093922 0.018029 */ price = 162 ==> vegie = Yes or cuisine = 058
 4. /* 0.093168 0.018510 */ price = 162 ==> cuisine = 221
 5. /* 0.092312 0.017548 */ price = 162 ==> vegie = Yes or cuisine = 142
 6. /* 0.091351 0.018269 */ price = 162 ==> vegie = Yes or cuisine = 009
 7. /* 0.089326 0.017308 */ price = 162 ==> cuisine = 229
 8. /* 0.088972 0.018990 */ price = 162 ==> cuisine = 058
 9. /* 0.088802 0.018269 */ price = 162 ==> cuisine = 142
10. /* 0.088764 0.018990 */ price = 162 ==> vegie = Yes or cuisine = 186 
Number of hypotheses considered: 10543
Number of hypotheses explored: 6667
\end{verbatim}
\caption{Tertius results}
 \label{fig:tv} 
\end{figure}

\addcontentsline{toc}{section}{Task vi}
\section*{Task vi}
\index{Task vi}
\begin{verbatim}
Determine the error that would be incurred by categorizing the restaurants based on
the continents they represent: Asia, Europe, Africa, North America, and South
America. For each continent, form rules to determine the outcome (which continent
they come from) based on other attributes such as price, atmosphere, quality of food,
etc.
\end{verbatim}
\subsection*{Results}
\begin{enumerate}
\item Assumptions 
\begin{enumerate}
\item The restaurants that contain geographical information features can be accurately categorized by continent.
\item The list of countries and continents is a complete list.
\item The restaurants that do not match the list of countries and continents cannot be categorized by continents and generate errors.
\end{enumerate} 
\end{enumerate}
The total number of restaurants: 4160.
The number of restaurants that contain geographical features: 3758.
$$Error=\frac{4160 - 3758}{4160} * 100\% = 9.66\%$$ 
Perl code to get the number of total restaurants and the number of restaurants that contain geographical features can bee seen in listing \hyperref[lst:vi]{\ref{lst:vi}}.
\newpage
\begin{appendices}
\section*{Perl Scripts}
\lstinputlisting[language=Perl,
frame=single,
caption={Task v Perl script},label=lst:v,captionpos=b,numbers=left]{../q5-Yun/q5.pl}
\lstinputlisting[language=Perl,
frame=single,
caption={Task vi Perl script},label=lst:vi,captionpos=b,numbers=left]{../q6-Yun/q6-prez.pl}
\section*{R Scripts}
\lstinputlisting[language=R,
frame=single,
caption={Task ii R script},label=lst:tii,captionpos=b,numbers=left]{../r/q2.R}
\lstinputlisting[language=R,
frame=single,
caption={Rule functions},label=lst:apendixRF,captionpos=b,numbers=left]{../r/ruleFunctions.R}
\lstinputlisting[language=R,
frame=single,
caption={Helper functions},label=lst:apendixHF,captionpos=b,numbers=left]{../r/helpers.R}
\end{appendices}
\end{document}